\documentclass{article}

\usepackage{graphicx}
\usepackage{hyperref}

\title{Machine Learning for Improving Surface-Layer-Flux Estimates}
\author{Dawit Hailu}

\begin{document}

\maketitle

\section{Introduction}

This document provides information about the Stability Function Hackathon.

\section{Document About the Hackathon}

The hackathon is focused on developing a machine learning model to predict the stability of the surface layer of the Earth's atmosphere. The document includes an introduction to the hackathon, information about the data source, tasks, and message from the organizers.

\section{Examples Available}

The hackathon includes examples of machine learning models for predicting the stability of the surface layer of the Earth's atmosphere.

\section{Data Source}

The hackathon uses data from the FluxNet project.

\section{Tasks}

The hackathon includes tasks such as producing histograms of input and output variables, scatter plots of correlations between variables, finding the maximum and minimum values of all variables, and training a machine learning model to predict the stability function.

\section{Visualization}

The hackathon includes visualizations of the total amount of data from each site and the mean values of parameters at each site.

\section{Writing Paper}

The hackathon includes instructions for writing a paper using Overleaf Latex.

\section{Balancing}

The hackathon includes instructions for assessing the balance of the data and developing a code to balance it.

\section{Training}

The hackathon includes instructions for training a machine learning model to predict the stability function.

\section{Message from Helena}

The hackathon includes a message from Helena, the organizer of the hackathon, about optimizing participation in the hackathon.

\end{document}
