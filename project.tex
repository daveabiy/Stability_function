\documentclass{article}

\usepackage{graphicx}
\usepackage{hyperref}

\title{Machine Learning for Improving Surface-Layer-Flux Estimates}
\author{Dawit Hailu}

\begin{document}

\maketitle

\section{Introduction}

This document provides information about the Stability Function Hackathon.

\section{Document About the Hackathon}

The hackathon is focused on developing a machine learning model to predict the stability of the surface layer of the Earth's atmosphere. The document includes an introduction to the hackathon, information about the data source, tasks, and message from the organizers.

\section{Theory}
Our theoretical path leads us from the climate modeling what we call the Primitive equations:
\newline
1. The equation explaining the horizontal motion of the atmosphere is given by
\begin{equation}
\frac{\partial u}{\partial t} + u \frac{\partial u}{\partial x} + v \frac{\partial u}{\partial y} + w \frac{\partial u}{\partial z} = - \frac{1}{\rho} \frac{\partial p}{\partial x} + fv
\end{equation}
where $u$ is the wind speed in the $x$ direction, $v$ is the wind speed in the $y$ direction, $w$ is the wind speed in the $z$ direction, $t$ is the time, $x$ is the $x$ coordinate, $y$ is the $y$ coordinate, $z$ is the $z$ coordinate, $\rho$ is the density, $p$ is the pressure, and $f$ is the Coriolis parameter.
\newline
2. The Hydrostatic equilibrium equation is given by
\begin{equation}
\frac{\partial p}{\partial z} = - \rho g
\end{equation}
where $p$ is the pressure, $z$ is the height, $\rho$ is the density, and $g$ is the gravitational acceleration.
\newline
3. The continuity equation is given by
\begin{equation}
\frac{\partial \rho}{\partial t} + \frac{\partial \rho u}{\partial x} + \frac{\partial \rho v}{\partial y} + \frac{\partial \rho w}{\partial z} = 0
\end{equation}
where $\rho$ is the density, $u$ is the wind speed in the $x$ direction, $v$ is the wind speed in the $y$ direction, $w$ is the wind speed in the $z$ direction, $t$ is the time, $x$ is the $x$ coordinate, $y$ is the $y$ coordinate, and $z$ is the $z$ coordinate.
\newline
4. The thermo-dynamic equation is given by
\begin{equation}
\frac{\partial \theta}{\partial t} + u \frac{\partial \theta}{\partial x} + v \frac{\partial \theta}{\partial y} + w \frac{\partial \theta}{\partial z} = 0
\end{equation}
where $\theta$ is the potential temperature, $u$ is the wind speed in the $x$ direction, $v$ is the wind speed in the $y$ direction, $w$ is the wind speed in the $z$ direction, $t$ is the time, $x$ is the $x$ coordinate, $y$ is the $y$ coordinate, and $z$ is the $z$ coordinate.

The above equations are used to determine the stability of the surface layer of the Earth's atmosphere. They are a funciton of wind speed, height, and potential temperature.
The variables we use for these functions:

\begin{list}{}{}
\item LWdown - Longwave radiation ($W m^{-2}$)
\item Precip - precipitation ($mm/hr$)
\item Psurf - Surface Pressure ($kPa$)
\item Qair - surface air specific humidity ($km km^{-1}$)
\item SWdown - Shortwave radiation ($W m^{-2}$)
\item Tair - Air temperature ($K$)
\item Ustar - Friction velocity ($m s^{-1}$) 
\item $u_{*}$ - surface friction wind speed, 
\item $k$ - von Karman constant, 
\item $z_{0}$ - initial height, 
\item $\psi$ - stability function, 
\item $Ri$ - Richardson number
\item $u$ - wind speed in the $x$ direction, $v$ - wind speed in the $y$ direction, $w$ - wind speed in the $z$ direction
\item Z0 - surface roughness ($m$)
\item Leaf Area Index - Leaf material in region
\item $\theta$ - potential temperature
\end{list}

We start from the thermo-dynamic energy equation, where $\theta$ is the potential temperature, T is the temperature, p is the pressure, $\rho$ is the density, $\vec{u}$ is the velocity field, and $\vec{g}$ is the gravitational acceleration.

\begin{equation}
\frac{D \theta}{D t} = \frac{\partial \theta}{\partial t} + \vec{u} \cdot \nabla \theta = \frac{\theta}{p} \frac{D p}{D t} - \frac{\theta}{\rho} \frac{D \rho}{D t} + \frac{\theta}{\rho} \vec{g} \cdot \vec{u}
\end{equation}
where $\frac{D}{D t}$ is the material derivative.

The Advention equation is given by
\begin{equation}
\frac{\partial \phi}{\partial t} + \vec{u} \cdot \nabla \phi = 0
\end{equation}
where $\phi$ is the quantity being advected, $\vec{u}$ is the velocity field, and $t$ is time.

The reason why the Advection Equation is important is because it is the simplest form of the continuity equation, which is a statement of conservation of mass. The continuity equation is given by
\begin{equation}
\frac{\partial \rho}{\partial t} + \nabla \cdot (\rho \vec{u}) = 0
\end{equation}
where $\rho$ is the density of the fluid.

The weather can be modeled using the Navier-Stokes equations, which are given by
\begin{equation}
\frac{\partial \vec{u}}{\partial t} + \vec{u} \cdot \nabla \vec{u} = -\frac{1}{\rho} \nabla p + \nu \nabla^2 \vec{u} + \vec{g}
\end{equation}
\begin{equation}
\frac{\partial \rho}{\partial t} + \nabla \cdot (\rho \vec{u}) = 0
\end{equation}
where $\vec{u}$ is the velocity field, $p$ is the pressure, $\rho$ is the density, $\nu$ is the kinematic viscosity, and $\vec{g}$ is the gravitational acceleration.

The discrete form of the thermo-dynamic energy equation is given by
\begin{equation}
\frac{\theta^{n+1} - \theta^n}{\Delta t} + \vec{u} \cdot \nabla \theta = \frac{\theta}{p} \frac{D p}{D t} - \frac{\theta}{\rho} \frac{D \rho}{D t} + \frac{\theta}{\rho} \vec{g} \cdot \vec{u}
\end{equation}
where $\theta$ is the potential temperature, $p$ is the pressure, $\rho$ is the density, $\vec{u}$ is the velocity field, and $\vec{g}$ is the gravitational acceleration.

The discrete form of the Advection equation is given by
\begin{equation}
\frac{\phi^{n+1} - \phi^n}{\Delta t} + \vec{u} \cdot \nabla \phi = 0
\end{equation}
where $\phi$ is the quantity being Advection, $\vec{u}$ is the velocity field, and $t$ is time.

The discrete form of the Navier-Stokes equations is given by
\begin{equation}
\frac{\vec{u}^{n+1} - \vec{u}^n}{\Delta t} + \vec{u} \cdot \nabla \vec{u} = -\frac{1}{\rho} \nabla p + \nu \nabla^2 \vec{u} + \vec{g}
\end{equation}
\begin{equation}
\frac{\rho^{n+1} - \rho^n}{\Delta t} + \nabla \cdot (\rho \vec{u}) = 0
\end{equation}
where $\vec{u}$ is the velocity field, $p$ is the pressure, $\rho$ is the density, $\nu$ is the kinematic viscosity, and $\vec{g}$ is the gravitational acceleration.

Every halving of the grid size increases the computational cost and memory requirement of the model by a factor of 8. Halving the grid size also leads to needing to halve the timestep, implying the computational cost of the model increases by a factor of 16.

\subsection{Stability Function}

The stability function is given by
\begin{equation}
f = \frac{g}{\theta} \frac{\partial \theta}{\partial z}
\end{equation}
where $f$ is the stability function, $g$ is the gravitational acceleration, $\theta$ is the potential temperature, and $z$ is the height.

The stability function is used to determine the stability of the surface layer of the Earth's atmosphere. 

\begin{equation}
u(z) = \frac{u_{*}}{k} \left[\ln \frac{z}{z_{0}} + \psi(z,z_{0},L)\right]
\end{equation}
where $u$ is the wind speed, $u_{*}$ is the surface friction wind speed, $z$ is the height, $z_{0}$ is the initial height, and $\psi$ is the stability function.

However in real life, there is a heat flux, which is given by
\begin{equation}
\frac{\partial \theta}{\partial t} = \frac{\partial \theta}{\partial z} \frac{\partial z}{\partial t} = \frac{\partial \theta}{\partial z} u
\end{equation}
where $\theta$ is the potential temperature, $z$ is the height, and $u$ is the wind speed.
And that is the reason why the stability function is used to determine the stability of the surface layer of the Earth's atmosphere.

\subsection{Richardson Number}
The Richardson number is given by 
\begin{equation}
Ri = \frac{g}{\theta} \frac{\partial \theta}{\partial z} / (\frac{\partial u}{\partial z})^2
\end{equation}
where $Ri$ is the Richardson number, $g$ is the gravitational acceleration, $\theta$ is the potential temperature, $z$ is the height, and $u$ is the wind speed.

Richardson number is used to determine the stability of the surface layer of the Earth's atmosphere. It is a function of vertical gradient of potential temperature, height, and wind speed.

We want the machine to learn the Richardson number from the data.
\section{Examples Available}

The hackathon includes examples of machine learning models for predicting the stability of the surface layer of the Earth's atmosphere.

\section{Data Source}

The hackathon uses data from the FluxNet project.

\section{Tasks}

The hackathon includes tasks such as producing histograms of input and output variables, scatter plots of correlations between variables, finding the maximum and minimum values of all variables, and training a machine learning model to predict the stability function.

\section{Visualization}

The hackathon includes visualizations of the total amount of data from each site and the mean values of parameters at each site.

\section{Writing Paper}

The hackathon includes instructions for writing a paper using Overleaf Latex.

\section{Balancing}

The hackathon includes instructions for assessing the balance of the data and developing a code to balance it.

\section{Training}

The hackathon includes instructions for training a machine learning model to predict the stability function.

\section{Message from Helena}

The hackathon includes a message from Helena, the organizer of the hackathon, about optimizing participation in the hackathon.

\end{document}
